\subsection{Vector Operations}

\subsubsection{Dot Product}
\begin{equation}
    g_{ij} = \e_i \cdot \e_j
\end{equation}
\begin{equation}
    \mathbf{A} \cdot \mathbf{B} = A^i B_j = g_{ij} A^i B^j \label{cl-dot}
\end{equation}

\subsubsection{Cross Product}
\begin{align}
    \mathbf{A} \times \mathbf{B} &= \jac \sum_k (A^i B^j - A^j B^i)\e^k \label{cl-cross-contra}\\
    &= \frac{1}{\jac} \sum_k (A_i B_j - A_j B_i)\e_k  \label{cl-cross-co}
\end{align}




\subsection{Del Operator in Curvilinear Coordinates}
Look for D'Haeseleer book \cite{dhaeseleer_flux_2012} page 35 for details.

Del operator for an arbitrary curvilinear coordinate frame is defined as,
\begin{equation}
    \nabla \equiv \nabla u^i \equiv \e^i \frac{\partial}{\partial u^i}
\end{equation}
\subsubsection{Gradient}
The gradient can be expressed as,
\begin{equation}
    \nabla \Phi = \sum_i \frac{\partial \Phi}{\partial u^i} \nabla u^i= \sum_i \frac{\partial \Phi}{\partial u^i} \e^i  \label{cl-grad}
\end{equation}

\subsubsection{Divergence}
The divergence can be expressed as,
\begin{equation}
    \nabla \cdot \mathbf{A} = \frac{1}{\jac} \sum_i \frac{\partial}{\partial u^i}(\jac A^i)  \label{cl-div}
\end{equation}
\subsubsection{Curl}
The curl can be expressed as,
\begin{align}
    \nabla \times \mathbf{A} &= \nabla \times (\sum_j A_j\e^j) \\
    \text{distribute curl} \hspace{1cm}&= \sum_j \left(A_j (\nabla \times \e^j) + \nabla A_j \times \e^j\right) \\
    \text{use $\nabla \times \e^j = $ 0 and definition of gradient} \hspace{1cm}&= \sum_j\sum_i \frac{\partial A_j}{\partial u^i} \e^i \times \e^j  \\
    \text{use cross product property eq. \ref{icrossj}} \hspace{1cm}&= \sum_j\sum_i \frac{\varepsilon^{ijk}}{\jac} \frac{\partial A_j}{\partial u^i} \e_k
\end{align}
\begin{equation}
    \nabla \times \mathbf{A} = \frac{1}{\jac} \sum_{k} \left( \frac{\partial A_j}{\partial u^i}  - \frac{\partial A_i}{\partial u^j}\right) \e_k   \label{cl-curl}
\end{equation}

% or
% \begin{align}
%     \nabla \times \mathbf{A} &= \nabla \times (\sum_j A^j\e_j) \\
%     &= \sum_j \left( A^j (\nabla \times \e_j) + \nabla A^j \times \e_j \right) \\
%     &= {\color{red}\text{JUST TRYING TO SEE IF THERE IS A WAY TO USE }} \\
%     &= {\color{red}\text{CONTRAVARIANT B DIRECTLY FOR CONTRAVARIANT J}}
% \end{align}

