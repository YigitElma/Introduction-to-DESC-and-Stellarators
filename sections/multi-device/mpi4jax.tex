\subsection{Installing MPI4JAX}
The installation on the website does not work as is, also for computing clusters in Princeton we already have C compilers for openmpi, so we will use them. The instructions are slightly complicated but as of February 18th 2025, it works. First, we will get openmpi from cluster modules, install jax with GPU support. Then, we will use the hacky way of installing mpi4py and mpi4jax as explained in the Princeton cluster page, note that that page use the trick only for mpi4py but we will use it for mpi4jax too, so, follow these instructions to revert the changes.

\begin{minted}[fontsize=\footnotesize]{bash}
module load anaconda3/2024.6
conda create -n desc-mpi python=3.12
conda activate desc-mpi
module load openmpi/gcc/4.1.6
# install jax
pip install -U jax['cuda12']
\end{minted}

For Python versions newer than 3.8, we have to use following as explained in \href{https://researchcomputing.princeton.edu/support/knowledge-base/mpi4py}{Princeton Research Computing page}.
\begin{minted}[fontsize=\footnotesize]{bash}
# install mpi4py
export MPICC=$(which mpicc)
cd /home/<YOUR_NETID>/.conda/envs/desc-mpi/compiler_compat
rm -f ld
ln -s /usr/bin/ld ld
pip install mpi4py --no-cache-dir
# move back to main folder, we don’t want to install it to /compiler_compat
cd ~
# To install `mpi4jax`, we need to grab the source code
# install mpi4jax
git clone https://github.com/mpi4jax/mpi4jax
cd mpi4jax
\end{minted}

Comment out a couple of lines in `setup.py` in mpi4jax directory (inside get\_sycl\_path()):
\begin{minted}[fontsize=\footnotesize]{python}
elif os.path.exists("/opt/intel/oneapi/compiler/latest/"):
	_sycl_path = "/opt/intel/oneapi/compiler/latest/"
\end{minted}

and then back in the command line,
\begin{minted}[fontsize=\footnotesize]{bash}
cd ..
python -c 'import mpi4jax' # check that it works
pip install -e .  # install mpi4jax to the environment, add it to the path
# Now that we are done with openMPI stuff, revert some changes previously made
cd /home/<NetID>/.conda/envs/desc-mpi/compiler_compat
rm -f ld
ln -s ../bin/x86_64-conda-linux-gnu-ld ld
# move back to main folder
cd ~
# install DESC as usual
\end{minted}

mpi4jax hello\_world example: \url{https://mpi4jax.readthedocs.io/en/latest/usage.html}

Note: I believe that we can skip cloning mpi4jax by setting some environment variable, it should basically use the proper compiler, maybe try to set “CMPLR\_ROOT”?
If we ever gonna make this a user feature, it needs to be less complicated as it can be. See \href{https://github.com/mpi4jax/mpi4jax/issues/245}{the issue} for future fixes.
