% --------------------------------------------------------------------
% @author       Kaya E. Unalmis
% @date         2023 July
% @command      lualatex rotational_transform_and_limit.tex
% @description  Math to compute the rotational transform in DESC.
% --------------------------------------------------------------------

% This file is edited by Yigit Gunsur Elmacioglu to include it 
% in the main text as a subsection. 

% \documentclass{article}

% ------------------------- package loading --------------------------
% \usepackage[
%     textwidth=12.5cm,
% ]{geometry}

% % math packages
% \usepackage[ISO]{diffcoeff}[=v4]
% \usepackage{mathtools}
% % load after mathtools (which loads amsmath)
% \usepackage{amsthm}
% % load after any other maths or font-related package
% \usepackage[
%     math-style=ISO,
%     warnings-off={mathtools-colon, mathtools-overbracket},
% ]{unicode-math}

% % load after most packages, in particular after font-related packages
% % makes things look better, apparently
% \usepackage{microtype}

% % almost always load last
% \usepackage{hyperref}
% \hypersetup{
% 	colorlinks=true,
% 	pdfdisplaydoctitle=true,
%     pdfauthor={Kaya E. Unalmis},
%     pdfpagemode=UseOutlines,
% 	pdftitle={Math to compute the rotational transform in DESC},
% }

% ------------------------ document settings -------------------------
% % elevate missing character warnings to errors
% \tracinglostchars=3

% % --- math commands ---
% % begin patch for amsthm bug
% \makeatletter
% \RenewDocumentEnvironment{proof}{O{\proofname}}{\par
%     \pushQED{\qed}%
%     \normalfont \topsep6\p@\@plus6\p@\relax
%     \trivlist
%     \item\relax
%         {\itshape
%     #1\@addpunct{.}}\hspace\labelsep\ignorespaces
% }{\popQED\endtrivlist\@endpefalse}
% \makeatother
% % end patch for amsthm bug
% \NewDocumentEnvironment{solution}{}{\begin{proof}[Solution]}{\end{proof}}

% % begin patch for unicode-math bug
% % https://tex.stackexchange.com/questions/547584/unicode-math-swallows-my-backslash
% \AtBeginDocument{\RenewDocumentCommand{\setminus}{}{\smallsetminus}}
% % end patch for unicode-math bug

% \theoremstyle{plain}
% \newtheorem{theorem}{Theorem}
% \newtheorem{lemma}{Lemma}
% \NewDocumentCommand{\lemmaautorefname}{}{Lemma}
% \theoremstyle{definition}
% \newtheorem{definition}{Definition}
% \NewDocumentCommand{\definitionautorefname}{}{Definition}
% \newtheorem{problem}{Problem}
% \NewDocumentCommand{\problemautorefname}{}{Problem}
% \theoremstyle{remark}
% \newtheorem*{remark}{Remark}

% % --- universal constants (not variables) ---
% \NewDocumentCommand{\e}{}{\symrm{e}}
% \NewDocumentCommand{\im}{}{\symrm{i}}
% \NewDocumentCommand{\pi}{}{\symrm{\pi}}

% % --- common math notation ---
% \DeclarePairedDelimiter{\abs}{\lvert}{\rvert}
% \DeclarePairedDelimiter{\norm}{\lVert}{\rVert}
% \DeclarePairedDelimiter{\ceil}{\lceil}{\rceil}
% \DeclarePairedDelimiter{\floor}{\lfloor}{\rfloor}
% \DeclarePairedDelimiter{\group}{\lparen}{\rparen}
% \DeclarePairedDelimiter{\groupbrack}{\lbrack}{\rbrack}
% % see mathtools package section 3.6 for explanation of the below set command
% % just to make sure it exists
% \ProvideDocumentCommand{\given}{}{}
% % can be useful to refer to this outside \set
% \NewDocumentCommand{\setSymbol}{o}{
%     \mathchoice{\:}{\:}{\,}{\,}\IfValueT{#1}{#1}\vert
%     \allowbreak
%     \mathchoice{\:}{\:}{\,}{\,}
%     \mathopen{}}
% \DeclarePairedDelimiterX{\set}[1]{\lbrace}{\rbrace}{%
%     \RenewDocumentCommand{\given}{}{\setSymbol[\delimsize]}
%     #1
% }
% % intervals
% \DeclarePairedDelimiterX{\closeint}[2]{\lbrack}{\rbrack}{#1, #2}
% \DeclarePairedDelimiterX{\openint}[2]{\lparen}{\rparen}{#1, #2}
% \DeclarePairedDelimiterX{\clopenint}[2]{\lbrack}{\rparen}{#1, #2}
% \DeclarePairedDelimiterX{\openclint}[2]{\lparen}{\rbrack}{#1, #2}
% % inner product
% \DeclarePairedDelimiterX{\innerp}[2]{\langle}{\rangle}{#1, #2}
% % sign function
% \DeclareMathOperator{\sign}{sign}
% % natural numbers from 1 to given argument
% \NewDocumentCommand{\nats}{m}{\symbb{N} \cap \closeint{1}{#1}}
% % integers from 0 to given argument
% \NewDocumentCommand{\ints}{m}{\symbb{Z} \cap \closeint{0}{#1}}

% % --- linear algebra ---
% % transpose
% \NewDocumentCommand{\trans}{m}{{#1}^{\mathsf{T}}}
% % Hermitian (conjugate) transpose
% \NewDocumentCommand{\herm}{m}{{#1}^{\mathsf{H}}}
% % image of argument (subset of codomain)
% \DeclareMathOperator{\image}{image}
% % rank of matrix
% \DeclareMathOperator{\rank}{rank}
% % dimension of kernel
% \DeclareMathOperator{\nullity}{nullity}
% % bold vectors
% \AtBeginDocument{\RenewDocumentCommand{\vec}{m}{\mathbf{#1}}}

% ------------------------- document content -------------------------
% \DeclarePairedDelimiter{\mean}{\langle}{\rangle}
% \title{Math to compute the rotational transform in DESC}
% \author{Kaya E. Unalmis}
% \date{August 29, 2023}

% \begin{document}

% \maketitle
% \tableofcontents

\subsection{Math to compute the rotational transform in DESC}
{\color{red} \textbf{NOTE:} This portion is reformatted, there might be missing symbols etc. Might be better read the original PDF}
\subsubsection{Terminology}
\begin{itemize}
	\item \(\mean{Q}_{\sqrt{g}} = (\oiint \dl{\theta} \dl{\zeta} \sqrt{g} \medspace Q) / (\oiint \dl{\theta} \dl{\zeta}  \sqrt{g})\) is the flux surface average of \(Q\).
	\item \(\mean{Q} = (\oiint \dl{\theta} \dl{\zeta} \medspace Q) / (\oiint \dl{\theta} \dl{\zeta})\) is the flux surface average without \(\sqrt{g}\) of \(Q\)
	\item \(I = \mean*{B_{\theta}} \) is the enclosed net toroidal current in tesla-meters
	\item \verb|data["I"]| \(= \mu_0\) \verb|data["current"]|$/2 \pi$
	\item \(\psi\) is the toroidal flux normalized by 2 $\pi$
	\item \(\rho\) is the flux surface label
	\item \(\diff{\psi}/{\rho}\) is \verb|data["psi_r"]| \(=\) \verb|params["Psi"]_r|$/ 2\pi$
	\item \(\lambda\) is the poloidal stream function
	\item \(\omega\) is the toroidal stream function
	\item \(\iota\) is the rotational transform normalized by $2 \pi$
\end{itemize}

\subsubsection{Derivation}
Any vector can be written as a contravariant coordinate vector dot product with covariant basis vectors.
\begin{equation}
	\mathbf{B} = B^{\rho} \mathbf{e_{\rho}} + B^{\theta} \mathbf{e_{\theta}} + B^{\zeta} \mathbf{e_{\zeta}}
\end{equation}
Assume nested flux surfaces.
\begin{equation}
	\mathbf{B} = B^{\theta} \mathbf{e_{\theta}} + B^{\zeta} \mathbf{e_{\zeta}}
\end{equation}
The components of the magnetic field are defined as
\begin{align}
	B^{\theta} & = \frac{\diff{\psi}/{\rho}}{\sqrt{g}} \group*{\iota - \diffp{\lambda}{\zeta}} \\
	B^{\zeta}  & = \frac{\diff{\psi}/{\rho}}{\sqrt{g}} \group*{1 + \diffp{\lambda}{\theta}}
\end{align}
Rearranging gives an equation for the rotational transform.
\begin{equation}
	\iota = \frac{\sqrt{g}}{\diff{\psi}/{\rho}} B^{\theta} + \diffp{\lambda}{\zeta}
\end{equation}
This suggests you can chain a relation between \(\iota\) and \(I\) through the common terms \(B^{\theta}\) and \(B_{\theta}\) in the equations for \(\iota\) and \(I\), respectively.
\begin{align}
	I & = \mean*{B_{\theta}}                                                                                    \\
	  & = \mean*{\mathbf{B} \cdot \mathbf{e_{\theta}}}                                                            \\
	  & = \mean*{\group*{B^{\theta} \mathbf{e_{\theta}} + B^{\zeta} \mathbf{e_{\zeta}}} \cdot \mathbf{e_{\theta}}}
\end{align}
Plugging in definitions for the coefficients of the metric tensor \(g\),
\begin{equation}
	I = \mean*{B^{\theta} g_{\theta \theta} + B^{\zeta} g_{\theta \zeta}}
\end{equation}
Surface average operations are additive homomorphisms.
\begin{equation}
	I = \mean*{B^{\theta} g_{\theta \theta}} + \mean*{B^{\zeta} g_{\theta \zeta}}
\end{equation}
Some rearranging eventually yields \(\iota\) as a function of \(I\).
\begin{align}
	\mean*{B^{\theta} g_{\theta \theta}}                                                                                                                       & = I - \mean*{B^{\zeta} g_{\theta \zeta}} \\
	\mean*{\frac{\diff{\psi}/{\rho}}{\sqrt{g}} \group*{\iota - \diffp{\lambda}{\zeta}} g_{\theta \theta}}                                                      & =                                        \\
	\mean*{\frac{\diff{\psi}/{\rho}}{\sqrt{g}} \iota g_{\theta \theta}} - \mean*{\frac{\diff{\psi}/{\rho}}{\sqrt{g}} \diffp{\lambda}{\zeta} g_{\theta \theta}} & =
\end{align}
For flux surface functions, surface average operations are the identity operation.
\begin{gather}
	\diff{\psi}{\rho} \group*{\iota \mean*{\frac{g_{\theta \theta}}{\sqrt{g}}} - \mean*{\diffp{\lambda}{\zeta} \frac{g_{\theta \theta}}{\sqrt{g}}}} = I - \mean*{B^{\zeta} g_{\theta \zeta}} \\
	\iota = \group*{\frac{I - \mean*{B^{\zeta} g_{\theta \zeta}}}{\diff{\psi}/{\rho}} + \mean*{\diffp{\lambda}{\zeta} \frac{g_{\theta \theta}}{\sqrt{g}}}} \bigg / \mean*{\frac{g_{\theta \theta}}{\sqrt{g}}}
	\label{eq:rotationalTransform}
\end{gather}

\subsubsection{Other arrangements}
The following are equivalent rearrangements of \eqref{eq:rotationalTransform}.
I am unsure which are more robust numerically.
Small magnitudes in the denominator will be bad.
Subtracting similar magnitude terms also results in floating point errors.
\begin{equation}
	\iota = \group*{\frac{I}{\diff{\psi}/{\rho}} + \mean*{\diffp{\lambda}{\zeta} \frac{g_{\theta \theta}}{\sqrt{g}} - \frac{B^{\zeta} g_{\theta \zeta}}{\diff{\psi}/{\rho}} }} \bigg / \mean*{\frac{g_{\theta \theta}}{\sqrt{g}}}
\end{equation}
Replacing \(B^{\zeta}\) gives
\begin{equation}
	\iota = \group*{\frac{I}{\diff{\psi}/{\rho}} + \mean*{\frac{\diffp{\lambda}{\zeta} g_{\theta \theta} - \group*{1 + \diffp{\lambda}{\theta}} g_{\theta \zeta}}{\sqrt{g}}}} \bigg / \mean*{\frac{g_{\theta \theta}}{\sqrt{g}}} \label{eq:surfaveogiota}
\end{equation}
Integral form
\begin{equation}
	\iota = \group*{4 \pi^2 I + \diff{\psi}{\rho} \oiint \dl{\theta} \dl{\zeta} \frac{\diffp{\lambda}{\zeta} g_{\theta \theta} - \group*{1 + \diffp{\lambda}{\theta}} g_{\theta \zeta}}{\sqrt{g}}} \bigg / \group*{\diff{\psi}{\rho} \oiint \dl{\theta} \dl{\zeta} \frac{g_{\theta \theta}}{\sqrt{g}}} \label{eq:integral:form}
\end{equation}
\begin{equation}
	\iota = \group*{4 \pi^2 \frac{I}{\diff{\psi}/{\rho}} + \oiint \dl{\theta} \dl{\zeta} \frac{\diffp{\lambda}{\zeta} g_{\theta \theta} - \group*{1 + \diffp{\lambda}{\theta}} g_{\theta \zeta}}{\sqrt{g}}} \bigg / \group*{\oiint \dl{\theta} \dl{\zeta} \frac{g_{\theta \theta}}{\sqrt{g}}} \label{eq:integral:form2}
\end{equation}
The term \(4 \pi^2 I\) is an additional term that supplements the zero current version of iota to account for the net toroidal current.
It can also be derived (not rigorously) by applying Ampere's law on a toroidal cross-section.
This is because \(\iota \cong \diff{\chi}/{\psi}\) (change in poloidal flux over toroidal flux) and Ampere's law shows that, for every thin slice of a toroidal cross-section, the toroidal current generates an extra poloidal flux of \(\int \dl{\theta} (\vec{B} \cdot \vec{e_{\theta}})\),
which is on average equal to \(2 \pi I \cong 2 \pi \mean{\vec{B} \cdot \vec{e_{\theta}}}\).
In \eqref{eq:surfaveogiota}, this factor of $2 \pi$ is hidden in \(\diff{\psi}/{\rho}\) because we normalized the latter by that same factor.
TODO: Can prove this rigorously.

\subsubsection{Axis limit of rotational transform}

We know that \(\psi \sim \rho^2\) and \(\vec{e_{\theta}} \to \vec{0}\) as \(\rho \to 0\).
It follows that
\begin{equation}
	\lim_{\rho \to 0} \norm{\mathbf{e_{\theta}}} = 0
	\implies
	\begin{cases}
		\lim_{\rho \to 0} g_{\theta \theta}                \\
		\lim_{\rho \to 0} \diffp{g_{\theta \theta}}/{\rho} \\
		\lim_{\rho \to 0} g_{\theta \zeta}                 \\
		\lim_{\rho \to 0} \sqrt{g}                         \\
		\lim_{\rho \to 0} I
	\end{cases} = 0
\end{equation}
\begin{equation}
	\begin{rcases}
		\lim_{\rho \to 0} \diffp{\norm{\mathbf{e_{\theta}}}}/{\rho} \\
		\lim_{\rho \to 0} \norm{\mathbf{e_{\zeta}}}                 \\
		\lim_{\rho \to 0} \norm{\mathbf{e_{\rho}}}
	\end{rcases} \neq 0 \implies
	\begin{cases}
		\lim_{\rho \to 0} \diffp[2]{g_{\theta \theta}}/{\rho} \\
		\lim_{\rho \to 0} \diffp{\sqrt{g}}/{\rho}
	\end{cases} \neq 0
\end{equation}

Computing the limits of the integrands in \eqref{eq:integral:form} shows that \eqref{eq:integral:form} is of the indeterminate form \(0 / 0\).
However, in that form, the denominator approaches zero more rapidly than the numerator, unless a non-basic combination of parameters is justified to be zero.
If that justification is made, then the same result obtained below is recovered after three additional applications of l'H\^opital's rule.

It turns out that the rearrangement that groups the toroidal current and the toroidal flux is more amenable to limit analysis than the others.
We know that the rotational transform must be finite at the axis.%
\footnote{This statement is equivalent to claiming \(\vec{B} \cdot (\vec{e_{\zeta}} \times \vec{e_{\rho}}) = 0\) at the magnetic axis.}
Since the denominator of \eqref{eq:surfaveogiota} tends to zero, it follows that \(\iota\) is finite only if its numerator tends to zero.
Hence, as defined in \eqref{eq:surfaveogiota}, \(\lim_{\rho \to 0} \iota\) is apparently of the indeterminate form \(0 / 0\).%
\footnote{%
	I assume interchanging the order of the limit and integral operations is valid in \eqref{eq:surfaveogiota}.

	A sufficient condition for interchanging is uniform convergence of the integrand.
	Other ways to prove validity of the interchange involve finding some region where the integrand is monotonic.
	I have failed to prove such properties.
	I think knowledge of relations that depend on the stellarator is required to deduce such properties.
	For example, uniform convergence of the simplest integrand in the denominator of \eqref{eq:surfaveogiota} requires finding a region, dependent on \(\epsilon\), where \(\norm{\mathbf{e_{\theta}}} < \epsilon \mathbf{e_{\theta}} / \norm{\mathbf{e_{\theta}}} \cdot \group{\mathbf{e_{\zeta}} \times \mathbf{e_{\rho}}}\) for all \(\epsilon > 0\).
	This task reduces to showing the covariant basis vectors are never orthogonal, with supremum of angle separation less than \(\pi / 2\).
	The alternative requires deducing if \(g_{\theta \theta} / \sqrt{g}\) is monotonic near \(\rho=0\).

	In any case, my justification for the interchange follows from our use of finite sums to compute the integrals.
	When the integrals are replaced with finite sums, the interchange is valid because the limit of each term to be summed exists.
	Perhaps more importantly, computation shows that the limit we derive here appears to be the continuous extension of \eqref{eq:surfaveogiota} to the \(\rho = 0\) axis.
	Since \(\iota\) is a physical quantity, we should expect \(\iota\) to be continuous over physical domains.
	This requires the limit to be a continuous extension.
}

This suggests we can find the limit of \(\iota\) without decomposing the basis functions.
With that goal, we define
\begin{align}
	\alpha & \cong 4 \pi^2 \frac{I}{\diff{\psi}/{\rho}}                                                                                                           \\
	\beta  & \cong \oiint \dl{\theta} \dl{\zeta} \frac{\diffp{\lambda}{\zeta} g_{\theta \theta} - \group*{1 + \diffp{\lambda}{\theta}} g_{\theta \zeta}}{\sqrt{g}} \\
	\gamma & \cong \oiint \dl{\theta} \dl{\zeta} \frac{g_{\theta \theta}}{\sqrt{g}}
\end{align}
Apply l'H\^opital's rule.
\begin{equation}
	\lim_{\rho \to 0} \iota = \lim_{\rho \to 0} \frac{\diff{\group{\alpha + \beta}}/{\rho}}{\diff{\gamma}/{\rho}}
	\label{eq:lhop1}
\end{equation}
The bounds of the integrals do not depend on \(\rho\), so we may differentiate under both integrals using the Leibniz integral rule.
\begin{align}
	\diff{\alpha}{\rho} & = \frac{4 \pi^2}{(\diff{\psi}/{\rho})^2} \group*{\diff{I}{\rho} \diff{\psi}{\rho} - I \diff[2]{\psi}{\rho}}                                                                                                                                                                                                                                                                                                                                                                                                                                                                                                      \\
	\diff{\beta}{\rho}  & = \oiint \dl{\theta} \dl{\zeta} \diffp*{\group*{\frac{\diffp{\lambda}{\zeta} g_{\theta \theta} - \group*{1 + \diffp{\lambda}{\theta}} g_{\theta \zeta}}{\sqrt{g}}}}{\rho}                                                                                                                                                                                                                                                                                                                                                                                                                                         \\
	                    & \begin{multlined}
		                      = \oiint \dl{\theta} \dl{\zeta} \left ( \frac{\diffp{\lambda}{\rho, \zeta} g_{\theta \theta} + \diffp{\lambda}{\zeta} \diffp{g_{\theta \theta}}{\rho} - \diffp{\lambda}{\rho, \theta} g_{\theta \zeta} - \group*{1 + \diffp{\lambda}{\theta}} \diffp{g_{\theta \zeta}}{\rho}}{\sqrt{g}} \right . \\
		                      \left . - \frac{\diffp{\lambda}{\zeta} g_{\theta \theta} - \group*{1 + \diffp{\lambda}{\theta}} g_{\theta \zeta}}{\sqrt{g}^2} \diffp{\sqrt{g}}{\rho}  \right )
	                      \end{multlined} \label{eq:muxi:numerator} \\
	\diff{\gamma}{\rho} & = \oiint \dl{\theta} \dl{\zeta} \diffp*{\group*{\frac{g_{\theta \theta}}{\sqrt{g}}}}{\rho}                                                                                                                                                                                                                                                                                                                                                                                                                                                                                                                        \\
	                    & = \oiint \dl{\theta} \dl{\zeta} \group*{\frac{\diffp{g_{\theta \theta}}{\rho} \sqrt{g} - g_{\theta \theta} \diffp{\sqrt{g}}{\rho}}{\sqrt{g}^2}} \label{eq:iotadenomderiv}
\end{align}

Now, if each of the following limits exist independently, and the limit in the denominator is nonzero,%
\footnote{We will later discover this precondition is true.}
then the following relation holds by the algebraic limit theorem.
\begin{equation}
	\lim_{\rho \to 0} \frac{\diff{\group{\alpha + \beta}}/{\rho}}{\diff{\gamma}/{\rho}} = \frac{\lim_{\rho \to 0} \diff{\alpha}/{\rho} + \lim_{\rho \to 0} \diff{\beta}/{\rho}}{\lim_{\rho \to 0} \diff{\gamma}/{\rho}}
	\label{eq:ALT}
\end{equation}
Apply l'H\^opital's rule twice to the derivative of the toroidal current supplement term.
\begin{equation}
	\lim_{\rho \to 0} \diff{\alpha}{\rho} = \frac{2 \pi^2}{(\diff[2]{\psi}/{\rho})^2} \group*{\diff[2]{I}{\rho} \diff[2]{\psi}{\rho} -\diff{I}{\rho} \diff[3]{\psi}{\rho}}
\end{equation}
Define \(\mu\) and \(\nu\) to be the numerator of the integrand in \eqref{eq:muxi:numerator} and \eqref{eq:iotadenomderiv}, respectively.
Continue with the assumption that the order of limit and integral operations can be interchanged.
\begin{align}
	\lim_{\rho \to 0} \diff{\beta}{\rho}  & = \oiint \dl{\theta} \dl{\zeta} \lim_{\rho \to 0} \frac{\mu}{\sqrt{g}^2} \\
	\lim_{\rho \to 0} \diff{\gamma}{\rho} & = \oiint \dl{\theta} \dl{\zeta} \lim_{\rho \to 0} \frac{\nu}{\sqrt{g}^2}
\end{align}
Both limits are of indeterminate form \(0 / 0\).
Recall that l'H\^opital's rule relies on Cauchy's generalized mean value theorem, which is a property proven unique to functions that map \(\symbb{R} \to \symbb{R}\).
Fortunately, some parts of l'H\^opital's rule have recently been generalized to multivariable functions that map \(\symbb{R}^n \to \symbb{R}\) where \(n \in \symbb{N}\).
Since the singularity is non-isolated and the zero set of the numerator contains the zero set of the denominator inside some small tube surrounding the magnetic axis, we may apply theorem 4, case 1 of the \href{https://doi.org/10.1080/00029890.2020.1793635}{multivariable form of l'H\^opital's rule}.
\begin{gather}
	\lim_{\rho \to 0} \frac{\mu}{\sqrt{g}^2} = \lim_{\rho \to 0} \frac{\diffp{\mu}/{\rho}}{\diffp{\sqrt{g}^2}/{\rho}} = \lim_{\rho \to 0} \frac{\diffp[2]{\mu}/{\rho}}{\diffp[2]{\sqrt{g}^2}/{\rho}} \\
	\lim_{\rho \to 0} \frac{\nu}{\sqrt{g}^2} = \lim_{\rho \to 0} \frac{\diffp{\nu}/{\rho}}{\diffp{\sqrt{g}^2}/{\rho}} = \lim_{\rho \to 0} \frac{\diffp[2]{\nu}/{\rho}}{\diffp[2]{\sqrt{g}^2}/{\rho}} \label{eq:toroidaldenomlimit}
\end{gather}
The limit of the denominator is
\begin{align}
	\diffp[2]{\sqrt{g}^2}{\rho}                   & = 2 \group*{\diffp{\sqrt{g}}{\rho}}^2 + 2 \sqrt{g} \diffp[2]{\sqrt{g}}{\rho} \\
	\lim_{\rho \to 0} \diffp[2]{\sqrt{g}^2}{\rho} & = 2 \group*{\diffp{\sqrt{g}}{\rho}}^2 \neq 0
\end{align}
Since the above limit is nonzero, we may compute each of the limits \(\diffp[2]{\mu}/{\rho}\) and \(\diffp[2]{\nu}/{\rho}\) independent of it.
\begin{multline}
	\diffp[2]{\mu}{\rho} = -\diffp[3]{\sqrt{g}}{\rho} \diffp{\lambda}{\zeta} g_{\theta \theta} + \diffp[3]{\sqrt{g}}{\rho} g_{\theta \zeta} \diffp{\lambda}{\theta} + \diffp[3]{\sqrt{g}}{\rho} g_{\theta \zeta} \\
	- \diffp{\lambda}{\zeta} \diffp[2]{\sqrt{g}}{\rho} \diffp{g_{\theta \theta}}{\rho} - g_{\theta \theta} \diffp[2]{\sqrt{g}}{\rho} \diffp{\lambda}{\rho, \zeta} + g_{\theta \zeta} \diffp[2]{\sqrt{g}}{\rho} \diffp{\lambda}{\rho, \theta} \\
	+ \diffp{g_{\theta \zeta}}{\rho} \group*{\diffp[2]{\sqrt{g}}{\rho} \group*{1 + \diffp{\lambda}{\theta}} - 3 \sqrt{g} \diffp[2, 1]{\lambda}{\rho, \theta}} \\
	+ \diffp{\sqrt{g}}{\rho} \left ( 2 \diffp{\lambda}{\rho, \zeta} \diffp{g_{\theta \theta}}{\rho} + \diffp{\lambda}{\zeta} \diffp[2]{g_{\theta \theta}}{\rho} + g_{\theta \theta} \diffp[2, 1]{\lambda}{\rho, \zeta} \right . \\
	\left . - 2 \diffp{g_{\theta \zeta}}{\rho} \diffp{\lambda}{\rho, \theta} - \group*{1 + \diffp{\lambda}{\theta}} \diffp[2]{g_{\theta \zeta}}{\rho} - g_{\theta \zeta} \diffp[2, 1]{\lambda}{\rho, \theta} \right ) \\
	+ 3 \sqrt{g} \diffp{\lambda}{\rho, \zeta} \diffp[2]{g_{\theta \theta}}{\rho} + 3 \sqrt{g} \diffp[2, 1]{\lambda}{\rho, \zeta} \diffp{g_{\theta \theta}}{\rho} + \sqrt{g} \diffp{\lambda}{\zeta} \diffp[3]{g_{\theta \theta}}{\rho} \\
	+ \sqrt{g} \diffp[3, 1]{\lambda}{\rho, \zeta} g_{\theta \theta} - 3 \sqrt{g} \diffp[2]{g_{\theta \zeta}}{\rho} \diffp{\lambda}{\rho, \theta} - \sqrt{g} \diffp[3]{g_{\theta \zeta}}{\rho} \diffp{\lambda}{\theta} \\
	- \sqrt{g} g_{\theta \zeta} \diffp[3, 1]{\lambda}{\rho, \theta} - \sqrt{g} \diffp[3]{g_{\theta \zeta}}{\rho}
\end{multline}
\begin{equation}
	\diffp[2]{\nu}{\rho} = \sqrt{g} \diffp[3]{g_{\theta \theta}}{\rho} + \diffp{\sqrt{g}}{\rho} \diffp[2]{g_{\theta \theta}}{\rho} - \diffp[2]{\sqrt{g}}{\rho}\diffp{g_{\theta \theta}}{\rho} - \diffp[3]{\sqrt{g}}{\rho} g_{\theta \theta}
\end{equation}
Take the limits.
\begin{multline}
	\lim_{\rho \to 0} \diffp[2]{\mu}{\rho} = \group*{1 + \diffp{\lambda}{\theta}} \diffp{g_{\theta \zeta}}{\rho} \diffp[2]{\sqrt{g}}{\rho} \\
	+ \group*{\diffp{\lambda}{\zeta} \diffp[2]{g_{\theta \theta}}{\rho} - 2 \diffp{\lambda}{\rho, \theta} \diffp{g_{\theta \zeta}}{\rho} - \group*{1 + \diffp{\lambda}{\theta}} \diffp[2]{g_{\theta \zeta}}{\rho}} \diffp{\sqrt{g}}{\rho}
\end{multline}
\begin{equation}
	\lim_{\rho \to 0} \diffp[2]{\nu}{\rho} = \diffp[2]{g_{\theta \theta}}{\rho} \diffp{\sqrt{g}}{\rho}
\end{equation}
Observe that \(\lim_{\rho \to 0} \diff{\gamma}/{\rho} \neq 0\) as required by \eqref{eq:ALT}.
\begin{equation}
	\lim_{\rho \to 0} \frac{\nu}{\sqrt{g}^2} = \frac{\diffp[2]{g_{\theta \theta}}/{\rho}}{2 \diffp{\sqrt{g}}/{\rho}} \neq 0 \label{eq:toroidaldenomlimithop}
\end{equation}

% \newgeometry{textwidth=15.6cm}
Collecting our results, we conclude:
\begin{equation}
	\lim_{\rho \to 0} \iota = \frac{\displaystyle 4 \pi^2 \frac{\diff[2]{I}{\rho} \diff[2]{\psi}{\rho} -\diff{I}{\rho} \diff[3]{\psi}{\rho}}{(\diff[2]{\psi}/{\rho})^2} + \oiint \dl{\theta} \dl{\zeta} \group*{\frac{\group*{1 + \diffp{\lambda}{\theta}} \diffp{g_{\theta \zeta}}{\rho} \diffp[2]{\sqrt{g}}{\rho} }{(\diffp{\sqrt{g}}/{\rho})^2} + \frac{\diffp{\lambda}{\zeta} \diffp[2]{g_{\theta \theta}}{\rho} - 2 \diffp{\lambda}{\rho, \theta} \diffp{g_{\theta \zeta}}{\rho} - \group*{1 + \diffp{\lambda}{\theta}} \diffp[2]{g_{\theta \zeta}}{\rho}}{\diffp{\sqrt{g}}/{\rho}}}}{\displaystyle \oiint \dl{\theta} \dl{\zeta} \frac{\diffp[2]{g_{\theta \theta}}/{\rho}}{\diffp{\sqrt{g}}/{\rho}}}
\end{equation}
Written with surface averages instead of integrals
\begin{equation}
	\lim_{\rho \to 0} \iota = \frac{\displaystyle \frac{\diff[2]{I}{\rho} \diff[2]{\psi}{\rho} -\diff{I}{\rho} \diff[3]{\psi}{\rho}}{(\diff[2]{\psi}/{\rho})^2} + \mean*{\frac{\group*{1 + \diffp{\lambda}{\theta}} \diffp{g_{\theta \zeta}}{\rho} \diffp[2]{\sqrt{g}}{\rho}}{(\diffp{\sqrt{g}}/{\rho})^2} + \frac{\diffp{\lambda}{\zeta} \diffp[2]{g_{\theta \theta}}{\rho} - 2 \diffp{\lambda}{\rho, \theta} \diffp{g_{\theta \zeta}}{\rho} - \group*{1 + \diffp{\lambda}{\theta}} \diffp[2]{g_{\theta \zeta}}{\rho}}{\diffp{\sqrt{g}}/{\rho}}}}{\displaystyle \mean*{\frac{\diffp[2]{g_{\theta \theta}}/{\rho}}{\diffp{\sqrt{g}}/{\rho}}}}
\end{equation}

\subsubsection{Rotational transform with toroidal stream function}
\begin{equation}
	\boxed{\iota = \frac{\displaystyle 4 \pi^2 \frac{I}{\diff{\psi}/{\rho}} + \oiint \dl{\theta} \dl{\zeta} \frac{\diffp{\lambda}{\zeta} g_{\theta \theta} - \group*{1 + \diffp{\lambda}{\theta}} g_{\theta \zeta}}{\sqrt{g}}}{\displaystyle \oiint \dl{\theta} \dl{\zeta} \frac{(1 + \diffp{\omega}{\zeta}) g_{\theta \theta} - \diffp{\omega}{\theta} g_{\theta \zeta}}{\sqrt{g}}}}
\end{equation}
The behavior of \(\diffp{\omega}/{\theta}\) near the magnetic axis should ensure that \(\gamma\) tends to zero.
The quantity \(\diff{\gamma}/{\rho}\) in \eqref{eq:iotadenomderiv} becomes
\begin{multline}
	\diff{\gamma}{\rho} = \oiint \dl{\theta} \dl{\zeta} \left( \frac{\diffp{\omega}{\rho, \zeta} g_{\theta \theta} + \group*{1 + \diffp{\omega}{\zeta}} \diffp{g_{\theta \theta}}{\rho} - \diffp{\omega}{\rho, \theta} g_{\theta \zeta} - \diffp{\omega}{\theta} \diffp{g_{\theta \zeta}}{\rho}}{\sqrt{g}}
	- \frac{\group*{1 + \diffp{\omega}{\zeta}} g_{\theta \theta} - \diffp{\omega}{\theta} g_{\theta \zeta}}{\sqrt{g}^2} \diffp{\sqrt{g}}{\rho} \right)
\end{multline}
Then \eqref{eq:toroidaldenomlimit} again requires multiple applications of the \(\symbb{R}^n \to \symbb{R}\) l'H\^opital's rule to evaluate.
Equation \eqref{eq:toroidaldenomlimithop} becomes
\begin{equation}
	\lim_{\rho \to 0} \frac{\nu}{\sqrt{g}^2} = \frac{\diffp{\omega}{\theta} \diffp{g_{\theta \zeta}}{\rho} \diffp[2]{\sqrt{g}}{\rho}}{2 (\diffp{\sqrt{g}}/{\rho})^2} + \frac{\group*{1 + \diffp{\omega}{\zeta}} \diffp[2]{g_{\theta \theta}}{\rho} - 2 \diffp{\omega}{\rho, \theta} \diffp{g_{\theta \zeta}}{\rho} - \diffp{\omega}{\theta} \diffp[2]{g_{\theta \zeta}}{\rho}}{2 \diffp{\sqrt{g}}/{\rho}} \neq 0
\end{equation}
Then the rotational transform at the magnetic axis is given by
\begin{equation}
	\lim_{\rho \to 0} \iota = \frac{\displaystyle 4 \pi^2 \frac{\diff[2]{I}{\rho} \diff[2]{\psi}{\rho} -\diff{I}{\rho} \diff[3]{\psi}{\rho}}{(\diff[2]{\psi}/{\rho})^2} + \oiint \dl{\theta} \dl{\zeta} \group*{\frac{\group*{1 + \diffp{\lambda}{\theta}} \diffp{g_{\theta \zeta}}{\rho} \diffp[2]{\sqrt{g}}{\rho} }{(\diffp{\sqrt{g}}/{\rho})^2} + \frac{\diffp{\lambda}{\zeta} \diffp[2]{g_{\theta \theta}}{\rho} - 2 \diffp{\lambda}{\rho, \theta} \diffp{g_{\theta \zeta}}{\rho} - \group*{1 + \diffp{\lambda}{\theta}} \diffp[2]{g_{\theta \zeta}}{\rho}}{\diffp{\sqrt{g}}/{\rho}}}}{\displaystyle \oiint \dl{\theta} \dl{\zeta} \group*{\frac{\diffp{\omega}{\theta} \diffp{g_{\theta \zeta}}{\rho} \diffp[2]{\sqrt{g}}{\rho}}{(\diffp{\sqrt{g}}/{\rho})^2} + \frac{\group*{1 + \diffp{\omega}{\zeta}} \diffp[2]{g_{\theta \theta}}{\rho} - 2 \diffp{\omega}{\rho, \theta} \diffp{g_{\theta \zeta}}{\rho} - \diffp{\omega}{\theta} \diffp[2]{g_{\theta \zeta}}{\rho}}{\diffp{\sqrt{g}}/{\rho}}}}
\end{equation}
Setting \(\omega\) and its derivatives to zero recovers the results without the toroidal stream function.
% \restoregeometry

\subsubsection{Axis limit of rotational transform derivatives}
The limits of the derivatives require similar work to compute.
Recall that as \(\rho \to 0\) so do \(\alpha + \beta\) and \(\gamma\).
For any \(x \in \set{\alpha + \beta, \gamma}\), \(\diff[n]{x}/{\rho}\) is finite and nonzero for all \(n \in \symbb{N}\).%
\footnote{Showing this requires \(n+1\) applications of the \(\symbb{R}^m \to \symbb{R}\) l'H\^opital's rule.}
Previously, we showed
\begin{align}
	\iota                   & = \frac{\alpha + \beta}{\gamma}                                                 \\
	\lim_{\rho \to 0} \iota & = \lim_{\rho \to 0} \frac{\diff{(\alpha + \beta)}/{\rho}}{\diff{\gamma}/{\rho}}
\end{align}
Now, the first derivative breaks from the \(0 / 0\) indeterminate form after two applications of the \(\symbb{R} \to \symbb{R}\) l'H\^opital's rule.
\begin{align}
	\diff{\iota}{\rho}                   & = \frac{\diff{(\alpha + \beta)}{\rho} \gamma - (\alpha + \beta) \diff{\gamma}{\rho}}{\gamma^2}                                                                                                                                                                                              \\
	\lim_{\rho \to 0} \diff{\iota}{\rho} & = \lim_{\rho \to 0} \frac{\diff[2]{(\alpha + \beta)}{\rho} \gamma - (\alpha + \beta) \diff[2]{\gamma}{\rho}}{2 \gamma (\diff{\gamma}/{\rho})}                                                                                                                                               \\
	                                     & = \lim_{\rho \to 0} \frac{\diff[3]{(\alpha + \beta)}{\rho} \gamma + \diff[2]{(\alpha + \beta)}{\rho} \diff{\gamma}{\rho} - \diff{(\alpha + \beta)}{\rho} \diff[2]{\gamma}{\rho} - (\alpha + \beta) \diff[3]{\gamma}{\rho}}{2 (\diff{\gamma}/{\rho})^2 + 2 \gamma (\diff[2]{\gamma}/{\rho})} \\
	                                     & = \lim_{\rho \to 0} \frac{\diff[2]{(\alpha + \beta)}{\rho} \diff{\gamma}{\rho} - \diff{(\alpha + \beta)}{\rho} \diff[2]{\gamma}{\rho}}{2 (\diff{\gamma}/{\rho})^2}
\end{align}
The second derivative eventually reduces to
\begin{align}
	\diff[2]{\iota}{\rho}                   & = \frac{\diff[2]{(\alpha + \beta)}{\rho} \gamma^2 - 2 \diff{(\alpha + \beta)}{\rho} \gamma \diff{\gamma}{\rho} + 2 (\alpha + \beta) \group*{\diff{\gamma}{\rho}}^2 - (\alpha + \beta) \gamma \diff[2]{\gamma}{\rho}}{\gamma^3}                                                                                                                            \\
	\lim_{\rho \to 0} \diff[2]{\iota}{\rho} & = \lim_{\rho \to 0} \frac{2 \diff[3]{(\alpha + \beta)}{\rho} \group*{\diff{\gamma}{\rho}}^2 - 3 \diff[2]{(\alpha + \beta)}{\rho} \diff{\gamma}{\rho} \diff[2]{\gamma}{\rho} + 3 \diff{(\alpha + \beta)}{\rho} \group*{\diff[2]{\gamma}{\rho}}^2 - 2 \diff{(\alpha + \beta)}{\rho} \diff{\gamma}{\rho} \diff[3]{\gamma}{\rho}}{6 (\diff{\gamma}/{\rho})^3}
\end{align}

% \end{document}
