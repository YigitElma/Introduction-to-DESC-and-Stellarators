% Shortcuts

% Please use the following shortcuts when you update a theory section or add a new one for consistency and 
% clarity of the document for future editing.

\newcommand{\todo}[1]{{\color{red}{#1}}}

\newcommand{\jac}{\sqrt{g}}
\newcommand{\B}{\mathbf{B}}
\newcommand{\J}{\mathbf{J}}
\newcommand{\e}{\mathbf{e}}

\newcommand{\dFluxRho}{\cfrac{\partial \psi_{T}}{\partial \rho}}

\newcommand{\dRr}{\cfrac{\partial R}{\partial \rho}}
\newcommand{\dRt}{\cfrac{\partial R}{\partial \theta}}
\newcommand{\dRz}{\cfrac{\partial R}{\partial \zeta}}

\newcommand{\dZr}{\cfrac{\partial Z}{\partial \rho}}
\newcommand{\dZt}{\cfrac{\partial Z}{\partial \theta}}
\newcommand{\dZz}{\cfrac{\partial Z}{\partial \zeta}}

\newcommand{\dLr}{\cfrac{\partial \lambda}{\partial \rho}}
\newcommand{\dLt}{\cfrac{\partial \lambda}{\partial \theta}}
\newcommand{\dLz}{\cfrac{\partial \lambda}{\partial \zeta}}

\newcommand{\Brho}{(B^\theta \e_\theta + B^\zeta \e_\zeta) \cdot \e_\rho}
\newcommand{\Btheta}{(B^\theta \e_\theta + B^\zeta \e_\zeta) \cdot \e_\theta}
\newcommand{\Bzeta}{(B^\theta \e_\theta + B^\zeta \e_\zeta) \cdot \e_\zeta}

\newcommand{\Bvector}{ \cfrac{1}{2\pi\jac}\dFluxRho\begin{bmatrix}
        \left(\iota - \dLz\right)\e_{\theta} + \left(1 + \dLt\right) \e_\zeta
    \end{bmatrix}
}

\newcommand{\Bvectoreval}{ \cfrac{1}{2\pi\jac}\dFluxRho\begin{bmatrix}
        \left(\iota - \dLz\right)\e_{\theta}|_{\rho,\zeta} + \left(1 + \dLt\right) \e_\zeta|_{\rho,\theta}
    \end{bmatrix}
}

\newcommand{\BsupT}{\cfrac{1}{2\pi\jac}\dFluxRho\left(\iota - \dLz\right)}
\newcommand{\BsupZ}{\cfrac{1}{2\pi\jac}\dFluxRho\left(1 + \dLt\right)}
\newcommand{\Jacobian}{R \left( \dRr  \dZt + \dRt  \dZr \right)}



% --- common math notation ---
\DeclarePairedDelimiter{\abs}{\lvert}{\rvert}
\DeclarePairedDelimiter{\norm}{\lVert}{\rVert}
\DeclarePairedDelimiter{\ceil}{\lceil}{\rceil}
\DeclarePairedDelimiter{\floor}{\lfloor}{\rfloor}
\DeclarePairedDelimiter{\group}{\lparen}{\rparen}
\DeclarePairedDelimiter{\groupbrack}{\lbrack}{\rbrack}

% see mathtools package section 3.6 for explanation of the below set command
% just to make sure it exists
\ProvideDocumentCommand{\given}{}{}
% can be useful to refer to this outside \set
\NewDocumentCommand{\setSymbol}{o}{
    \mathchoice{\:}{\:}{\,}{\,}\IfValueT{#1}{#1}\vert
    \allowbreak
    \mathchoice{\:}{\:}{\,}{\,}
    \mathopen{}}
\DeclarePairedDelimiterX{\set}[1]{\lbrace}{\rbrace}{%
    \RenewDocumentCommand{\given}{}{\setSymbol[\delimsize]}
    #1
}
% intervals
\DeclarePairedDelimiterX{\closeint}[2]{\lbrack}{\rbrack}{#1, #2}
\DeclarePairedDelimiterX{\openint}[2]{\lparen}{\rparen}{#1, #2}
\DeclarePairedDelimiterX{\clopenint}[2]{\lbrack}{\rparen}{#1, #2}
\DeclarePairedDelimiterX{\openclint}[2]{\lparen}{\rbrack}{#1, #2}
% inner product
\DeclarePairedDelimiterX{\innerp}[2]{\langle}{\rangle}{#1, #2}
% sign function
\DeclareMathOperator{\sign}{sign}
% natural numbers from 1 to given argument
\NewDocumentCommand{\nats}{m}{\symbb{N} \cap \closeint{1}{#1}}
% integers from 0 to given argument
\NewDocumentCommand{\ints}{m}{\symbb{Z} \cap \closeint{0}{#1}}

% --- linear algebra ---
% transpose
\NewDocumentCommand{\trans}{m}{{#1}^{\mathsf{T}}}
% Hermitian (conjugate) transpose
\NewDocumentCommand{\herm}{m}{{#1}^{\mathsf{H}}}
% image of argument (subset of codomain)
\DeclareMathOperator{\image}{image}
% rank of matrix
\DeclareMathOperator{\rank}{rank}
% dimension of kernel
\DeclareMathOperator{\nullity}{nullity}
% % bold vectors
% \AtBeginDocument{\RenewDocumentCommand{\vec}{m}{\mathbf{#1}}}
\DeclarePairedDelimiter{\mean}{\langle}{\rangle}

\newtcbox{\class}[1][]{on line,
    colback=plasmalightgreen,
    coltext=white,
    boxrule=0pt,
    arc=4pt, % Controls the roundness of corners
    boxsep=1pt,
    left=2pt,
    right=2pt,
    top=1pt,
    bottom=1pt,
    #1}
    
\newtcbox{\obj}[1][]{on line,
    colback=plasmagreen,
    coltext=white,
    boxrule=0pt,
    arc=4pt, % Controls the roundness of corners
    boxsep=1pt,
    left=2pt,
    right=2pt,
    top=1pt,
    bottom=1pt,
    #1}

\newtcbox{\opt}[1][]{on line,
    colback=plasmadarkgreen,
    coltext=white,
    boxrule=0pt,
    arc=4pt, % Controls the roundness of corners
    boxsep=1pt,
    left=2pt,
    right=2pt,
    top=1pt,
    bottom=1pt,
    #1}
\newcommand{\function}[1]{{\color{blue}{#1}}}
\newcommand{\attribute}[1]{{\color{red}{#1}}}
\newcommand{\dx}{\Delta x}
\newcommand{\dc}{\Delta c}